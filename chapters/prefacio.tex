\chapter*{Prefacio}
\addcontentsline{toc}{chapter}{Prefacio} % Add the preface to the table of contents as a chapter

Este libro no pretende ser un decálogo sobre como se deberían abordar los patrones de diseño durante el desarrollo de software, actualmente existe una gamma muy extensa de recursos físicos y digitales que cumplen esta función a cabalidad. No obstante, el objetivo de este libro es orientar a estudiantes y profesionales en el área de la ingeniería de software que buscan aprender, comprender e implementar patrones de diseño en sus actividades al momento de desarrollar y diseñar software, asi como un conjunto de herramientas que le permitan desarrollar software de mayor calidad.

Por otro lado, este libro tampoco espera llegar a ser un reporte técnico detallado que busca describir de manera excesivamente compleja conceptos propios de programación orientada a objetos o de algoritmia. En este sentido, es recomendable (aunque no excluyente) que cualquiera que pretenda leer este libro cuente con conocimientos básicos en programación y diseño orientado a objetos que le permitan entender sin mayor inconvenientes conceptos como composición, agregación, herencia o polimorfismo, que son aspectos clave al momento de hablar sobre patrones en el desarrollo de software.

Este libro busca ofrecer una explicación lo suficientemente clara para que el lector pueda comprender e interiorizar los patrones de diseño mediante esquemas teóricos y ejemplos prácticos que representen soluciones a problemas comunes y fáciles de entender. Adicionalmente, este libro se apoya en las mejores prácticas y las recomendaciones hechas por otros autores que ya han escrito libros abordando esta temática (¡Después de todo no se trata de reinventar la rueda!).

El estudio de patrones de diseño no requiere conocimientos extensos en programación orientada a objetos o grandes habilidades como desarrollador. Todas las soluciones presentadas en este libro pueden ser desarrolladas con un poco de conocimiento en programación orientada a objetos y algo de esfuerzo. Finalmente, este libro es el resultado de varios años de observación en los cuales tuve la oportunidad de pasar por varias universidades, empresas y cursos con la motivación de aprender a detalle la definición y uso de patrones diseño. El objetivo final de este recurso es ampliar la perspectiva de aquellos lectores que busquen apoyar su proceso actual de desarrollo, ofreciendo un conjunto de herramientas conceptuales y prácticas que le permitan realizar diseños y desarrollos flexibles, escalables, reutilizables y fáciles de entender.

\begin{flushright}
	\textit{Carlos Orozco}
\end{flushright}
