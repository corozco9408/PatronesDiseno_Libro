\chapter*{Prólogo}
\addcontentsline{toc}{chapter}{Prologo} % Add the preface to the table of contents as a chapter

En la actualidad, existe una extensa gamma de documentación física y digital creada por expertos en la ingeniería de software que desde su experiencia nos han ofrecido material invaluable para comprender y aplicar un conjunto de mejores prácticas para desarrollar software de alta calidad. No obstante, a pesar que el campo de la arquitectura y el diseño de software han avanzado a pasos agigantados durante las ultimas tres décadas y hoy practicamente cualquier arquitecto de software conoce sobre patrones de diseño, este conocimiento no se ha replicado de manera que cualquier estudiante o profesional en el área de la ingeniería de software tenga dominio sobre estos temas que son cruciales en la implementación de software con altos estándares de calidad, llegando al punto de escuchar a estudiantes o profesionales que recurren a implementaciones muy elaboradas y rebuscadas para solucionar problemas que podrían abordarse de manera transparente aplicando un patrón conocido.

Las arquitecturas y modelos orientados a objetos cuentan con bases sólidas y bien estructuradas para abstraer soluciones adaptadas a cada dominio en particular. Sin embargo, como todos llegamos a notar en alguna ocasión mera casualidad o gracias a la experiencia, el desarrollo de software cae constantemente en la necesidad solucionar problemas recurrentes o que siguen un patrón establecido. La importancia de los patrones radica principalmente en la necesidad de solucionar problemas comunes en el desarrollo orientado a objetos que otros antes de nosotros (con gran astucia) lograron identificar y catalogar para establecer un conjunto de lineamientos y caminos a seguir para evitar reinventar la rueda y enfocar nuestros esfuerzos en el desarrollo de aspectos específicos relacionados al domino de nuestro problema.

Autores como Erich Gamma, Richad Helm, Ralph Jhonson y Jhon Vlissides (conocidos como el GoF o La banda de cuatro, del inglés "The Gang of Four") fueron los precursores en la definición de patrones de diseño y nos dieron la oportunidad de abordar este apasionante tema desde una perspectiva teórica y práctica. Otros como Kathy Sierra y Elisabeth Freeman nos ofrecieron una perspectiva didáctica y llena de ejemplos para abordar los patrones de diseño mediante un espectro amplio y bien estructurado de ejemplos. Asimismo, hay otros muchos autores en el área que han escrito material de primera calidad como los mencionados anteriormente, asi como otros libros que no mencionaré que pueden resultar confusos o muy dificiles de comprender para el lector. Dicho esto, este libro busca encontrar un punto de encuentro que busque presentar la terminología asociada al área de interés, a la par de ofrecer un soporte teórico sólido, e ir mas allá mediante ejemplos fáciles de comprender a través de una estructura clara y apoyada en las mejores prácticas propuestas por otros autores y expertos en la materia que ya han arado el camino que estamos por abordar.