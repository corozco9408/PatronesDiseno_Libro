\chapter*{Guía para lectores}
\addcontentsline{toc}{chapter}{Guía para lectores} % Add the preface to the table of contents as a chapter

Este libro se divide en tres partes, el Capítulo 1 es una introducción que busca dar una visión general sobre los aspectos mas relevantes sobre la definición y el uso de patrones de diseño en el desarrollo de software. Los Capítulos 2, 3 y 4 presenta el listado de patrones de acuerdo a su clasificación (creacionales, estructurales y de comportamiento) siguiendo la siguiente estructura: (i) inicialmente se presenta un resumen teórico de cada patrón para contextualizar al lector, (ii) presentación del modelo teórico que sustenta el patrón incluyendo los pro y contra en su implementación apoyados en modelos UML y esquemas teóricos fáciles de entender y (iii) Descripción de ejemplos prácticos pensados para comprender cada patrón a partir de su aplicación en un contexto real o utilizando analogías fáciles de entender para el lector. Finalmente, el Capítulo 5 se agregó como complemento para que el lector conozca algunos de los antipatrones mas conocidos en el diseño de aplicaciones y que normalmente no son abordados en libros de esta temática (¡conoce a tu enemigo y aprende a evitarlo!)

Este material no fue diseñado para ser abordado como una receta o un camino lineal que debe seguirse de inicio a fin. Por fortuna para todos nosotros, a pesar que los patrones interactuan entre sí y están definidos en diferentes categorías, pueden ser abordados uno a uno de manera independiente, así un lector interesado en aprender, o repasar patrones específicos lo puede hacer de manera transparente. Si no tienes mucha experiencia en el diseño orientado a objetos puedes seguir el libro en el orden establecido gracias a que la caracterización de patrones existentes va desde lo mas sencillo hasta los patrones que tienen un grado mayor de complejidad.